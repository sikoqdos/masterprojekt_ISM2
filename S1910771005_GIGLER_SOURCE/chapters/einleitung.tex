\setlength{\parindent}{0em} 

%**********************************************************************************
% Kapitel Einleitung
%**********************************************************************************
\chapter{Einleitung}
\label{cha:einleitung}
Das Thema IT-Sicherheit wird für Unternehmen immer essentieller. Speziell der Banken-Sektor stellt ein lukratives Ziel für Angreifer dar, da Banken als als \glqq{}Verteilzentren\grqq{} weltweiter Finanztransaktionen dienen. In den IT-Systemen von Banken werden Geldtransaktionen aus aller Welt gespeichert und Kundendaten verarbeitet. 
Laut einer Studie der \glqq{}Boston Consulting Group\grqq{} (BCG\footnote{https://www.bcg.com}) treffen Cyberattacken Finanzdienstleister 300-Mal häufiger als Unternehmen aus anderen Sparten. Trotz dieser Tatsache sind viele Finanzdienstleister nicht genug auf Cyberangriffe und deren Folgen vorbereitet. 
Unterschiedliche Faktoren wie der steigende Konkurrenzdruck und die Etablierung von nicht-traditionellen Marktteilnehmern und Fintech-Unternehmen führen dazu, dass der Bereich der IT-Sicherheit bei Finanzdienstleistern oft von Einsparungen betroffen ist. \autocite{Zakrzewsk2019} 
\bigbreak
Ein Cyberangriff oder ein Ausfall von IT-Infrastrukturkomponenten kann zu einem erheblichen finanziellen und reputationstechnischen Schaden für Finanzdienstleister und deren Partner führen. Ein weiterer möglicher Schaden eines Cyberangriffs ist der Verlust des Vertrauens der Kunden in das betroffene Finanzinstitut. Aus diesem Grund existieren für Finanzinstitute weitreichende Vorschriften betreffend der Sicherheit von IT-Systemen. \cite{EY2021} 
\bigbreak
Das Risiko eines Cyberangriffs, etwa auf die Infrastruktur eines Finanzdienstleisters, wird um einen weiteren Faktor verschärft. Es kann vorkommen, dass es Finanzdienstleistern nicht möglich ist, eine eigene IT-Infrastruktur zu betreiben. Aus diesem Grunde können unterschiedliche Aufgaben an Dritte ausgelagert werden um Kosten zu sparen und somit einen Wettbewerbsvorteil zu lukrieren. Anstatt beispielsweise die notwendige IT-Infrastruktur selbst zu betreiben, nötige Prozesse zu etablieren und Mitarbeiter zu beschäftigen, werden diese Aufgaben von spezialisierten Dienstleistern erbracht. Durch diese Auslagerung entsteht für den Finanzdienstleister eine neue Art von Risiko. Neben dem tatsächlichen Ausfall von Hard- oder Software oder einem Angriff auf die IT-Infrastruktur wird der gesamte Dienstleister zu einem Risiko. Dieses Risiko besteht, wenn der Dienstleister seine Leistungen nicht oder nicht ausreichend erbringt, oder selbst Opfer eines Cyberangriffs wird. \autocite{Presse2019} 
\bigbreak
Aufgrund der steigenden Relevanz von IT-Sicherheit im Zuge von Auslagerungen hat die \glqq{}Europäische Bankenaufsichtsbehörde\grqq{} (EBA\footnote{https://www.eba.europa.eu}) im Jahr 2019 eine Leitlinie zum Umgang mit Auslagerungsrisiken veröffentlicht (EBA/GL/2019/02) \autocite{eba_leitlinien_konvergenzkriterien}. Parallel dazu wird auf europäischer Ebene an einer Harmonisierung regulatorischer Anforderungen an die IT-Sicherheit von Finanzunternehmen in Europa gearbeitet. Die \glqq{}Europäische Aufsichtsbehörde für das Versicherungswesen und die betriebliche Altersversorgung\grqq{} (EIOPA\footnote{https://www.eiopa.europa.eu}) veröffentlichte gemeinsam mit der EBA und der \glqq{}Europäischen Wertpapier- und Marktaufsichtsbehörde\grqq{} (ESMA\footnote{https://www.esma.europa.eu}) eine Stellungnahme mit dem Titel \glqq{}Joint Advice on the need for legislative improvements relating to Information and Communication Technoloy risk management requirements\grqq{} \autocite{european_banking_authority_2019}. 
In dieser Stellungnahme werden konkreten Maßnahmen zur Harmonisierung und Konvergenz von Anforderungen an die Sicherheit der Informations- und Kommunikationstechnologie von Finanzunternehmen vorgeschlagen \autocite{Bafin2019}. 

%**********************************************************************************
% Kapitel Problemstellung und Zielsetzung
%**********************************************************************************
\section{Problemstellung und Zielsetzung}
\label{cha:zielsetzung}
Aufgrund der steigenden Gefahr von Cyber-Angriffen im Finanzdienstleistungssektor ist es äußerst wichtig allgemein geltende Anforderungen an IT-Sicherheit zu etablieren. Aufgrund der Komplexität des Themas IT-Sicherheit als solches und dem stetigen Wandel im Bereich Informationstechnik, stellen Aufsichtsbehörden nur Rahmenwerke zur Verfügung. Detaillierte technische und fachliche Anforderungen und damit einhergehende zu implementierende Maßnahmen werden in den Anforderungen häufig nicht oder nur oberflächlich vorgegeben. Aufsichtsbehörden fordern von Unternehmen etwa eine IT-Sicherheitsrichtlinie zu etablieren, diese umzusetzen und die Erfüllung der IT-Sicherheitsrichtlinie regelmäßig zu überprüfen. Die genauen Anforderungen an eine IT-Sicherheitsrichtlinie sind jedoch oft nicht im Detail spezifiziert, werden von den Finanzdienstleistern nach bestem Wissen und Gewissen subjektiv interpretiert und daraus entsprechende Erkenntnisse und Maßnahmen abgeleitet. Die tatsächliche technische und fachliche Umsetzung dieser Maßnahmen stellt viele Finanzdienstleister vor eine große Hürde. Eine weitere Herausforderung für Finanzdienstleister besteht darin den Überblick über aktuell geltende IT-Sicherheitsanforderungen an das eigene Unternehmen zu behalten. Je nach Standort und Tätigkeitsfeld des Unternehmens sind verschiedene Aufsichtsbehörden für ein Unternehmen zuständig und somit unterschiedliche regulatorische Anforderungen zu erfüllen. Es existiert beispielsweise kein allgemein gültiger Anforderungskatalog für IT-Sicherheit für Banken in Österreich. Welche Anforderungen im Bereich IT-Sicherheit zu erfüllen sind hängt von der jeweils prüfenden Aufsichtsbehörde ab.
\bigbreak
Die vorliegende Masterarbeit dient als Überblick für regulatorische Anforderungen an die IT-Sicherheit von Banken in Österreich und als Basis für die Implementierung von technischen Maßnahmen zur Erfüllung dieser Anforderungen. Im Zuge der Ausarbeitung dieser Masterarbeit wird untersucht, welche technischen Anforderungen an die IT-Sicherheit einer Bank in Österreich bestehen und ob diese mit Hilfe von Best-Practice-Ansätzen erfüllt werden können. Um dieser Frage nachgehen zu können werden im Zuge der Ausarbeitung aktuell geltende technische Anforderungen an IT-Sicherheit von Aufsichtsbehörden erhoben, in deren Zuständigkeit österreichische Bank-Institute fallen. Die erhobenen Anforderungen werden in weiterer Folge miteinander in Verbindung gebracht und kategorisiert. Im Anschluss daran werden mögliche technische Maßnahmen zur Umsetzung dieser Anforderungen abgeleitet. Schlussendlich wird untersucht welche Best-Practice-Ansätze sich für die Umsetzung der geforderten Maßnahmen eignen. Im Zuge der Ausarbeitung wird des Weiteren auf das Thema eingegangen, wie die Umsetzung der einzelnen Maßnahmen gemessen und auditiert werden kann. Abschließend wird analysiert ob Angriffsvektoren und Cyberbedrohungen bestehen, die mit den Anforderungen und abgeleiteten Maßnahmen nicht adressiert werden. 
\bigbreak
Fazit der Masterarbeit bildet eine Anforderungsmatrix mit aktuell gültigen Anforderungen, abgeleiteten technischen Maßnahmen, einer Übersicht über das jeweilig adressierte Risiko und einer Übersicht, mit welchen Best-Practice-Ansätzen diese Maßnahmen umgesetzt und deren Erfüllung gemessen werden kann.

%**********************************************************************************
% Kapitel Forschungsfrage und Methodik
%**********************************************************************************
\section{Forschungsfrage und Methodik}
\label{ch:Forschungsfrage}
Forschungsfrage: Können geltende technische IT-Sicherheitsanforderungen an eine Bank in Österreich auf Basis von Best-Practice-Ansätzen erfüllt werden?\\\\
Nebenfrage 1: Welche regulatorischen, technischen Anforderungen und Vorgaben im Bereich IT-Sicherheit bestehen für ein Bank-Institut in Österreich?\\
Nebenfrage 2: Welche technisch erforderlichen Maßnahmen können aus diesen technischen Anforderungen und Vorgaben abgeleitet werden?\\
Nebenfrage 3: Welche Best-Practice-Ansätze eignen sich für die Umsetzung der erforderlichen Maßnahmen?\\
Nebenfrage 4: Wie kann gemessen werden, ob die technischen Anforderungen und Vorgaben erfüllt werden?
\bigbreak
Die Grundlage der Masterarbeit bildet eine Recherche über Aufsichtsbehörden mit Zuständigkeit im Finanzdienstleistungssektor. Der Fokus wurde im Zuge der Recherche noch nicht auf Bankunternehmen in Österreich gelegt um eine umfassende Sicht auf das Thema zu erhalten. So werden etwa auch Aufsichtsbehörden anderer Ausprägungen von Finanzdienstleistungsgesellschaften, wie etwa Versicherungen oder reinen Wertpapiergesellschaften, durchgeführt. Anschließend wird eine Literaturrecherche bezogen auf aktuell gültige technische IT-Sicherheitsanforderungen von Aufsichtsbehörden erhoben. Sowohl die Recherche nach relevanten Aufsichtsbehörden als auch die Literaturrecherche nach technischen IT-Sicherheitsanforderungen wird auf Basis allgemein zugänglicher Informationen im Internet und durch Befragung von Angestellten aus den Bereichen Governance und IT-Sicherheit eines österreichischen Finanzdienstleistungsunternehmen durchgeführt.
Im nächsten Schritt werden die erhobenen technischen IT-Sicherheitsanforderungen in Bezug auf ihre Gültigkeit für eine Bank in Österreich geprüft. Dies dient dazu, die Zuständigkeit der analysierten Aufsichtsbehörden zu klären und damit die Relevanz der jeweiligen Anforderungen an Banken in Österreich abgrenzen zu können.
Um die Vollständigkeit der analysierten Aufsichtsbehörden und Anforderungen zu prüfen wird ein Peer Review mit drei Angestellten aus dem Bereich IT-Sicherheit und IT-Governance einer Bank in Österreich abgehalten. Das Ergebnis des Peer Reviews wird in Kapitel \ref{peer_review} behandelt. 
\bigbreak
Die öffentlich zugänglichen Dokumente der relevanten Aufsichtsbehörden werden in weiterer Folge analysiert und jeweils ein kurzer Überblick in Form eines Steckbriefs erstellt. Die Steckbriefe sind in Kapitel \ref{Übersicht_Literatur} ersichtlich. Auf Basis der gewonnenen Erkenntnisse wird eine Anforderungsmatrix aufgebaut und die technischen Anforderungen an IT-Sicherheit aus den Dokumenten darin festgehalten. Der Aufbau der Anforderungsmatrix wird in Kapitel \ref{Anforderungsmatrix} behandelt. 
Die analysierten Anforderungen werden im Anschluss auf Basis ihrer möglichen Umsetzbarkeit in \glqq{}Technische Anforderungen\grqq{} und \glqq{}Organisatorische Anforderungen\grqq{} aufgeteilt.
Für die Technischen-Anforderungen werden mögliche Maßnahmen nach aktuellem Stand der Technik analysiert und Best-Practice-Ansätze recherchiert. Sowohl die Analyse möglicher Maßnahmen als auch die Recherche von Best-Practice-Ansätzen sind auf Basis frei zugänglicher Informationen im Internet und Gesprächen mit Angestellten aus dem Bereich IT-Infrastruktur und IT-Sicherheit eines Bank-Instituts in Österreich erhoben worden. Die Maßnahmen werden in weiterer Folge gruppiert und Best-Practice-Ansätze abgeleitet. Im Zuge der Analyse wird aufgezeigt, wie der Ansatz zur Umsetzung der technischen Maßnahme beitragen kann. Als Qualitätskontrolle der gewonnenen Erkenntnisse wird ein Abgleich mit der \glqq{}Cyber Defense Matrix\grqq{} des \glqq{}Open Web Application Security Project\grqq{} (OWASP\footnote{https://owasp.org}) durchgeführt \autocite{owasp_cyber_defense_matrix}. Auf Basis diese Abgleichs kann erkannt werden, ob die Anforderungen der Aufsichtsbehörden alle gängigen Angriffsvektoren beziehungsweise Cyber-Bedrohungen abdecken und ob die analysierten Best-Practice-Ansätze legitim sind. 

%**********************************************************************************
% Kapitel Abgrenzung
%**********************************************************************************
\section{Abgrenzung}
\label{cha:abgrenzung}
Im Zuge der Ausarbeitung dieser Masterarbeit werden nur Aufsichtsbehörden analysiert, in deren Zuständigkeitsbereich österreichische Bankinstitute fallen. Betrachtet werden nur Anforderungen die für die Gründung einer Bank in Österreich relevant und vorgeschrieben sind. Grundlegende IT-Sicherheitsanforderungen an Unternehmen, wie beispielsweise die Erfüllung der Datenschutzgrundverordnung (DSGVO), werden in der Anforderungsmatrix nicht beachtet \autocite{datenschutzbehörde}. Die erhobenen IT-Sicherheitsanforderungen können sich zum Teil mit allgemein gültigen IT-Sicherheitsanforderungen an Unternehmen außerhalb des Finanzdienstleistungssektors decken. Im Zuge der Ausarbeitung werden nur IT-Sicherheitsanforderungen analysiert, die technisch umgesetzt werden können. Die für die Ausarbeitung dieser Masterarbeit analysierten Aufsichtsbehörden und die verwendeten IT-Sicherheitsanforderungen sind in Kapitel \ref{Übersicht_Literatur} aufgelistet. 