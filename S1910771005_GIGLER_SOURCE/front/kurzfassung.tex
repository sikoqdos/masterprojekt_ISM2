\chapter{Kurzfassung}
Aufgrund der steigender Bedrohung von Cyberangriffen auf IT-Systemen stellen Aufsichtsbehörden verstärkt Vorgaben an die IT-Sicherheit von Unternehmen. Ein Cyberangriff oder ein Ausfall von IT-Infrastrukturkomponenten kann einen erheblichen finanziellen und reputationstechnischen Schaden für Unternehmen darstellen. Vor allem die Sicherheit von Banken und anderer Finanzdienstleister stehen verstärkt im Fokus von Aufsichtsbehörden. Für Finanzinstitute gilt es einerseits die technische Sicherheit der eigenen IT-Umgebung zu erhöhen um Angriffe zu vereiteln und andererseits den unterschiedlichen Anforderungen von Aufsichtsbehörden nachzukommen. Die vorliegende Arbeit beschäftigt sich mit Anforderungen von Aufsichtsbehörden an die technische IT-Sicherheit von Banken in Österreich. Im Zuge dessen werden für die Ausarbeitung relevante Richtlinien erhoben und die Vollständigkeit der untersuchten Richtlinien mittels Peer Reviews sichergestellt. Die Richtlinien werden in weiterer Folge und auf technisch umzusetzende Sicherheitsanforderungen hin analysiert. Diese Sicherheitsanforderungen werden kategorisiert und mögliche technische Umsetzungen abgeleitet. Es wird der Frage nachgegangen, ob sich diese technischen Sicherheitsanforderungen auch mittels bereits etablierter Best-Practice-Ansätze umsetzen lassen. Ziel der vorliegenden Arbeit ist es einen Überblick über technische IT-Sicherheitsanforderungen an Banken in Österreich zu geben, die jeweilig adressierten Themenbereiche zu erörtern und mögliche Maßnahmen zur Umsetzung darzulegen.